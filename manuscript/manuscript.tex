\documentclass[12pt]{article}

\usepackage{amsmath}
\usepackage[margin = 1in]{geometry}
\usepackage[colorlinks=true, citecolor=blue]{hyperref}
\usepackage{hyperref}

\title{Using Machine Learning to Estimate the Changes of Success of College Basketball Transfers}
\author{Miles Kee}

\begin{document}
\maketitle

\begin{abstract}
When the NCAA decided to eliminate the rule of athletes needing to sit out a season after transferring due to the COVID-19 pandemic, it created a frenzy in the college basketball world. Record numbers of players transferring were seen immediately after this ruling. Since then, the NCAA has decided to make this elimination permanent. This, combined with student athletes newfound ability to profit off their name, image, and likeness, has turned the college basketball offseason into a scene that is more akin to the free agency period in professional sports. Over 1500 players entered the transfer portal during the 2023 offseason. College coaches have began rebuilding their entire teams using the transfer portal instead of the traditional program-building methods of recruiting well and emphasizing development. New St. John's coach Rick Pitino was hired in March, and by early May, he had already brought in eight new transfers while losing another eight to the transfer portal. This type of roster turnover has become routine in the college game, especially when a coach is fired or hired. 
\end{abstract}

\end{document}